\documentclass[letterpaper,12pt]{article}
\author{Shawn Ma}
\date{\today}
\title{Circle1 1}

\usepackage[pdfusetitle]{hyperref}
\usepackage{amsmath}
\usepackage{amssymb}
\usepackage{asymptote}
%\usepackage{floatrow}
\usepackage{geometry}
%\usepackage{multicol}

\geometry{left=2cm,right=2cm,top=1.5cm,bottom=1.5cm}
\begin{document}
\setlength{\parindent}{0pt}
%\renewcommand{\familydefault}{\sfdefault}

\begin{enumerate}

\item In the semicircle below, AB is the diameter and O is the center. D is the midpoint of $\overset{\frown}{AC}$.
$DE\perp{AB}$. Prove that $2\cdot{DE}=AC$.


\begin{asy}
    import geometry;
    size(0,3.5cm);
    point A=(-5,0), B=(5,0);
    arc a = arcsubtended(A, B, 90);
    point C=relpoint(a, 1/4);
    point D=relpoint(a,2.5/4);
    point EE=projection(line(A,B))*D;
    draw(a);
    draw(A--B);
    draw(A--C);
    draw(D--EE);

    label("$A$",A,S);
    label("$B$",B,S);
    label("$C$",C,NE);
    label("$D$",D,N);
    label("$E$",EE,S);
    dot("$O$",origin,S);
\end{asy}

\item Two circles share the same center at $O$. A line intersects them at A,B,C,D. Assume the
radius of the big circle is $5$ and the smaller circle is $3$, what is $AD\cdot{BD}$?

\begin{asy}
    import geometry;
    size(0,5cm);
    point A=(-5,0), B=(5,0);
    circle a = circle(A, B);
    circle b = circle((-3,0),(3,0));
    draw(a);draw(b);
    line k=line((-1,2),(1,2));
    point[] pa =intersectionpoints(k,a);
    point[] pb =intersectionpoints(k,b);
    label("$A$",pa[0],W);
    label("$B$",pa[1],E);
    label("$C$",pb[0],NW);
    label("$D$",pb[1],NE);
    draw(pa[0]--pa[1]);
    dot("$O$",origin,S);

\end{asy}

% 3
\item $BC$ is the diameter of $\odot{O}$. The radius of the circle is $\cfrac{3}{2}$. $BH\perp{AD}$. $\overline{AB}=2$. What is $\cfrac{BH}{BD}$?

\begin{asy}
 import geometry;
    size(0,5cm);
    point A=(sqrt(5),0), B=(sqrt(5),2),C=(0,0);
    circle a = circle(A, B, C);
    draw(a);
    draw(A--B--C);
    point D=relpoint(a, 1/2);
    point H=projection(line(D,A))*B;
    draw(H--B);
    markrightangle(B,H,D);
    draw(B--D--A);
    label("$A$",A,SE);
    label("$B$",B,NE);
    label("$C$",C,SW);
    label("$D$",D,W);
    label("$H$",H,S);
    dot("$O$",a.C,S);

\end{asy}

%4 
\item In the circle below, AB is the diameter. Line $CD$ intersects the circle at $E$, $F$. $AC$ and $BD$ are perpendicular to $CD$.
Prove that $CE$ = $FD$.

\begin{asy}
    import geometry;
    size(0,5cm);
    point A=(-5,4), B=(5,1);
    circle a = circle(A, B);
    point C=(-5,-2), D=(5,-2);
    line k=line(C,D);
    point[] pa=intersectionpoints(a,k);

    draw(a);
    draw(A--B--D--C--A);
    dot("$O$",a.C,S);
    label("$A$",A,W);
    label("$B$",B,E);
    label("$C$",C,S);
    label("$D$",D,S);
    label("$E$",pa[0],S);
    label("$F$",pa[1],S);
    markrightangle(A,C,D);
    markrightangle(B,D,C);
\end{asy}

\item AB is the diameter of $\odot{O}$. BC is tangent to $\odot{O}$ at B. BD intersects $\odot{O}$ at E and AB at F.
AF=2, DF=3, EF=6. Find $\overline{CE}$.

\begin{asy}
    import geometry;
    size(0,5cm);
    point A=(0,5), B=(0,-5);
    circle a = circle(A, B);
    point C=(7,-5), D=relpoint(a, 5.7/18);
    line k=line(C,D);
    point[] pa=intersectionpoints(a,k);
    point F=intersectionpoint(k, line(A,B));

    draw(a);
    draw(A--B--C--D);
    dot("$O$",a.C,W);
    label("$A$",A,N);
    label("$B$",B,S);
    label("$C$",C,S);
    label("$D$",D,NW);
    label("$E$",pa[0],E);
    label("$F$",F,E);
\end{asy}

\item ABCD is a parallelogram. The $\odot{O}$ is the circumscribed circle of A,B,C. CD is tangent to the circle.
AB=4, BE=5. Find $\overline{DE}$.

\begin{asy}
    import geometry;
    size(0,5cm);
    point A=(-2,0), B=(2,0), C=(0,5), D=(-4,5);
    circle a = circle(A, B, C);

    draw(a);
    draw(A--B--C--D--A);

    point[] p=intersectionpoints(a,line(A,D));
    draw(p[1]--B);
    dot("$O$",a.C,W);
    label("$A$",A,S);
    label("$B$",B,S);
    label("$C$",C,N);
    label("$D$",D,NW);
    label("$E$",p[1],W);
\end{asy}

\item ABCD is a square. Semicircle $\odot{O}$ has AB as its diameter. DF tangents $\odot{O}$ at E. BF=4. Find $\cfrac{AF}{DF}$ and the length of BE.

\begin{asy}
    import geometry;
    size(0,5cm);
    point A=(-2,0), B=(2,0), C=(2,4), D=(-2,4);
    arc a = arcsubtended(A, B, 90);
    circle c=circle(A,B);
    draw(a);
    draw(A--B--C--D--A);

    line t=tangents(c,D)[1];
    point F=intersectionpoint(t, line(A,B));
    draw(B--F--D);
    point EE=intersectionpoints(a,t)[0];
    draw(B--EE);
    dot("$O$",c.C,S);
    label("$A$",A,S);
    label("$B$",B,S);
    label("$C$",C,N);
    label("$D$",D,NW);
    label("$E$",EE,NE);
    label("$F$",F,S);
\end{asy}

\end{enumerate}

\end{document}
