\documentclass[letterpaper,12pt]{article}
\author{Shawn Ma}
\date{\today}
\title{Right triangle 1}

\usepackage[pdfusetitle]{hyperref}
\usepackage{amsmath}
\usepackage{amssymb}
\usepackage{asymptote}
%\usepackage{floatrow}
\usepackage{geometry}
%\usepackage{multicol}

\geometry{left=2cm,right=2cm,top=1.5cm,bottom=1.5cm}
\begin{document}
\setlength{\parindent}{0pt}
%\renewcommand{\familydefault}{\sfdefault}

\begin{enumerate}

\item In the right $\triangle$ blow, $\angle{C}=90^\circ$. $a, b, c$ are the length of BC, AC and AB. $a:c=3:5$, $b=32$, find $a, c$.

\begin{asy}
    import geometry;
    size(4cm);
    triangle t=triangleabc(3,4,5);
    draw(t);label(t);
    draw("$a$",t.B--t.C);
    draw("$b$",t.A--t.C);
    draw("$c$",t.A--t.B);
\end{asy}

\item A triangle has 3 sides of $a,b,c$. They satisfy $|a-\sqrt{7}|+\sqrt{b-5}+(c-4\sqrt{2})^2=0$. Is it a right triangle?

% 3
\item In the right triangle below, $\angle{A}=90^\circ$. BD=10, what is AC?

\begin{asy}
    import geometry;
    size(5cm);
    triangle t=triangle((0,0),(sqrt(3),0),(0,1));
    draw(t);label(t);
    point D=(1,0);
    draw(t.C--D);
    draw("$D$",D, S);
    markangle("$45^\circ$",t.C,D,t.A);
    markangle("$30^\circ$",t.C,t.B,t.A);
    
\end{asy}

%4 
\item In the right $\triangle{OXY}$ below, $\angle{O}=90^\circ$. M, N are the midpoints of OX, OY.
XN=19 and YM=22, find XY.

\begin{asy}
    import geometry;
    size(5cm);
    point O=(0,0);
    point X=(4,0);
    point Y=(0,3);
    point M=(O+X)/2;
    point NN=(O+Y)/2;
    draw(O--X--Y--O);
    draw(X--NN);
    draw(Y--M);
    label("$O$",O,S);
    label("$X$",X,S);
    label("$Y$",Y,N);
    label("$M$",M,S);
    label("$N$",NN,W);
\end{asy}

\item The diagram below shows a rectangle with side lengths $4$ and $8$ and a square with side length $5$. Three vertices of the square lie on three different sides of the rectangle, as shown. What is the area of the region inside both the square and the rectangle?

\begin{asy}
    import geometry;
    size(5cm);
    filldraw((4,0)--(8,3)--(8-3/4,4)--(1,4)--cycle,mediumgray);
    draw((0,0)--(8,0)--(8,4)--(0,4)--cycle,linewidth(1.1));
    draw((1,4)--(4,0)--(8,3)--(5,7)--(1,4),linewidth(1.1));
    draw((1,0)--(1,4),dashed);
    label("$4$", (8,2), E);
    label("$8$", (4,0), S);
    label("$5$", (3,11/2), NW);
    //draw((1,.35)--(1.35,.35)--(1.35,0),linewidth(1.1));
\end{asy}



\pagebreak
\item In triangle ABC, $CA=CB$. $AB=\sqrt{3}-1$ and $\angle{C}=30^\circ$. Find BC.

\begin{asy}
    import geometry;
    size(5cm);
    triangle t=triangleabc(2,2,sqrt(3)-0.8);
    draw(t);label(t);
\end{asy}

\item In the triangle ABC, AB=6, BC=8. Find AC in each of the following cases:
\begin{enumerate}
    \item $\angle{B}=30^\circ$
    \item $\angle{B}=45^\circ$
    \item $\angle{B}=135^\circ$
\end{enumerate}

\item In triangle RST, $RS=13, ST=14, RT=15$.
\begin{enumerate}
    \item Find the length of the height from R to ST. (Hint: Heron's thereom, \\ $S=\sqrt{s(s-a)(s-b)(s-c)}$ where $s=\cfrac{a+b+c}{2}$).
    \item M is the midpoint of ST, find the length of RM.
\end{enumerate}

\begin{asy}
    import geometry;
    size(5cm);
    triangle t=triangleabc(15,13,14);
    draw(t);label(t,LA="$S$",LB="$T$",LC="$R$");
    point m =foot(t.AB);
    draw(t.C--m);
    label("$H$",m,S);
    m=midpoint(t.AB);
    draw(t.C--m);
    label("$M$",m,S);
\end{asy}
\end{enumerate}

\end{document}
