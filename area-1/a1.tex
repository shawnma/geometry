\documentclass[letterpaper,12pt]{article}
\author{Shawn Ma}
\date{\today}
\title{Area, practice 7}

\usepackage[pdfusetitle]{hyperref}
\usepackage{amsmath}
\usepackage{amssymb}
\usepackage{asymptote}
%\usepackage{floatrow}
\usepackage{geometry}
%\usepackage{multicol}

\geometry{left=2cm,right=2cm,top=1.5cm,bottom=1.5cm}
\begin{document}

\section*{Basic ideas}
\begin{enumerate}
    \item The midline splits a triangle into two equal parts.
    \item The areas of two triangles are the same if they have same base and height.
    \item The diaganal line splits a parallelagram into two equal parts.
    \item If the bases of two triangles are the same, the ratio of their areas is the same as their heights.
\end{enumerate}
\section*{Problems}
\setlength{\parindent}{0pt}

\begin{enumerate}

\item In the $\triangle{ABC}$ below, $D$ is the midpoint of $BC$
and $E$ is the midpoint of $AD$. $S_{\triangle{ABC}}=1$, what is the area of $\triangle{ABE}$?

\begin{asy}
import markers;
import math;
import geometry;
size(5cm);
pair A,B,C,D,EE,F,G,H;
A=(1,4);
B=(-3,0);
C=(3,0);
D=(0,0);
line l=line(A,D);
EE=midpoint(l);
draw(A--B--C--A);
draw(A--F);
draw(EE--B);
label("$A$",A,N);
label("$B$",B,S);
label("$C$",C,S);
label("$D$",D,S);
label("$E$",EE,E);
\end{asy}



\item In the diagram below, $BE\parallel AD \parallel CF$.
\begin{enumerate}
\item Prove that $S_{\triangle{AEF}}=S_{\triangle{ABC}}$
\item Prove that $S_{\triangle{EDF}}=2S_{\triangle{ABC}}$
\end{enumerate}

\begin{asy}
import markers;
import math;
import geometry;
size(7cm);
pair A,B,C,D,EE,F,G,H;
A=(1,2);
B=(-3,0);
C=(3,0);
D=(0.5,0);
line l=line(A,D);
line lb=parallel(B,l);
line lc=parallel(C,l);
EE=intersectionpoint(lb,line(A,C));
F=intersectionpoint(lc,line(A,B));
draw(A--B--C--A);
draw(D--EE--F--cycle);
draw(C--EE);
draw(B--EE--F--C);
draw(A--D);
draw(B--F);
label("$A$",A,N);
label("$B$",B,S);
label("$C$",C,S);
label("$D$",D,S);
label("$E$",EE,W);
label("$F$",F,E);

\end{asy}

\item In the $\triangle{ABC}$ below, $D$ is the midpoint of $BC$. $CE\perp{AD}, BF\perp{AD}$. \\Prove that $CE=BF$ WITHOUT using congurency.

\begin{asy}
import markers;
import math;
import geometry;
size(5cm);
pair A,B,C,D,EE,F,G,H;
A=(1,4);
B=(-3,0);
C=(3,0);
D=(0,0);
line l=line(A,D);
EE=projection(l)*C;
F=projection(l)*B;
draw(A--B--C--A);
draw(A--F);
draw(EE--C);
draw(F--B);
label("$A$",A,N);
label("$B$",B,S);
label("$C$",C,S);
label("$D$",D,SE);
label("$E$",EE,W);
label("$F$",F,S);
markrightangle(B,F,D);
markrightangle(C,EE,A);
\end{asy}

\item In the equilateral $\triangle{ABC}$ below, $h$ is the height.
P is a random point in $\triangle{ABC}$. \\ $PE\perp{AC}, PF\perp{AB}, PD\perp{BC}$.
Prove that $PD+PE+PF=h$.

\begin{asy}
import markers;
import math;
import geometry;
size(5cm);
pair A,B,C,D,EE,F,G,H,P;
A=(0, 4*sqrt(3));
B=(-4,0);
C=(4,0);
P=(-1,2.2);
D=projection(line(B,C))*P;
EE=projection(line(A,C))*P;
F=projection(line(A,B))*P;
draw(A--B--C--A);
draw(P--F);draw(P--EE);draw(P--D);
draw(P--A,dashed); draw(P--B, dashed); draw(P--C,dashed);
//line h = perpendicular(A, line(B,C));
H=projection(B,C)*A;
draw("$h$", A--H,E,0.6*red);
//draw("$h$",h);
label("$A$",A,N);
label("$B$",B,S);
label("$C$",C,S);
label("$D$",D,S);
label("$E$",EE,E);
label("$F$",F,W);
label("$P$",P,SE);
markrightangle(P,F,B);
markrightangle(P,EE,C);
markrightangle(P,D,C);
\end{asy}

\item $\bigstar$ In the $\triangle{ABC}$ below, $AD$ bisects $\angle{BAC}$. Prove that $\dfrac{AB}{AC}=\dfrac{BD}{DC}$.

\begin{asy}
import markers;
import math;
import geometry;
size(5cm);
pair A,B,C,D,EE,F,G,H;
A=(1,4);
B=(-4,0);
C=(3,0);
line b = bisector(line(A,B),line(A,C));
D=intersectionpoint(b,line(B,C));
draw(A--B--C--A);
draw(A--D);
label("$A$",A,N);
label("$B$",B,S);
label("$C$",C,S);
label("$D$",D,S);
\end{asy}

\item $\bigstar$ In the trapezoid $ABCD$ below, $AB\parallel CD$. $M$ is the midpoint of $BC$.
\\Prove that $S_{\triangle{AMD}}=\dfrac{1}{2}S_{ABCD}$

\begin{asy}
import markers;
import math;
import geometry;
size(5cm);
pair A,B,C,D,EE,F,G,H;
A=(-3,0);
B=(4,0);
C=(3,4);
D=(-1,4);
EE=midpoint(line(B,C));
draw(A--B--C--D--cycle);
draw(A--EE--D);
label("$A$",A,S);
label("$B$",B,S);
label("$C$",C,N);
label("$D$",D,N);
label("$M$",EE,E);

\end{asy}

\end{enumerate}


\end{document}