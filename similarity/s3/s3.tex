\documentclass[letterpaper,12pt]{article}
\author{Shawn Ma}
\date{\today}
\title{Similar triangle 3}

\usepackage[pdfusetitle]{hyperref}
\usepackage{amsmath}
\usepackage{amssymb}
\usepackage{asymptote}
%\usepackage{floatrow}
\usepackage{geometry}
%\usepackage{multicol}

\geometry{left=2cm,right=2cm,top=1.5cm,bottom=1.5cm}
\begin{document}
\setlength{\parindent}{0pt}
%\renewcommand{\familydefault}{\sfdefault}

\begin{enumerate}

\item In $\triangle{ABC}$ below, D is a point on $\overline{BC}$. How many possibilities are there for a point E such that E is on one side
of $\triangle{ABC}$ and the triangle formed by D, E plus another vertex of $\triangle{ABC}$ is similar with $\triangle{ABC}$?

\begin{asy}
    import geometry;
    size(4cm);
    triangle t=triangleabc(6,8,7);
    draw(t);label(t);

    point D=(2*t.B+t.C)/3;
    dot(D);
    label("$D$",D,E);

\end{asy}

\item In isosceles $\triangle{ABC}$ below, $\angle{A}=36^\circ$. BD bisects $\angle{ABC}$, prove that $\triangle{ABC}\sim\triangle{BCD}$.

\begin{asy}
    import geometry;
    size(4cm);
    point B=(0,0), C=(4,0);
    line l1=rotate(72)*line(B,C);
    line l2=rotate(-72,C)*line(B,C);
    point A=intersectionpoint(l1,l2);
    triangle t=triangle(A,B,C);
    draw(t);label(t);

    line l3=rotate(36)*line(B,C);
    point D=intersectionpoint(l3,t.AC);
    draw(B--D);
    label("$D$",D,E);
\end{asy}

\item $\spadesuit$ In Diagram below, $CD=AE$. Prove that $DF\cdot{BC}=AB\cdot{EF}$.

\begin{asy}
    import geometry;
    size(5cm);
    triangle t=triangleabc(6,8,7);
    draw(t);label(t, LA="");
    label("$A$",t.A,S);

    point D=(1.5*t.B+t.C)/2.5;
    point EE=(-length(t.C-D),0);
    draw(D--EE);draw(t.A--EE);
    label("$E$",EE,S);
    label("$D$",D,E);
    
    line p=parallel(D,t.AC);
    point K=intersectionpoint(p,t.AB);
    draw(D--K, dashed);
    draw("$K$",K,S);

    point F=intersectionpoint(t.AC,line(D,EE));
    label("$F$",F,NW);
\end{asy}

\item In Diagram below, $\angle{ACB}=90^\circ, DM\perp{AB}$. M is the midpoint of AB. Prove that
\begin{enumerate}
    \item $MC^2=MD\cdot ME$.
    \item $\cfrac{CE^2}{CD^2}=\cfrac{ME}{MD}$
\end{enumerate}

\begin{asy}
    import geometry;
    size(5cm);
    triangle t=triangleabc(4,3,5);
    draw(t);label(t);
    
    point M=midpoint(t.AB);
    label("$M$",M,S);
    line p = perpendicular(M,t.AB);
    point D=intersectionpoint(t.AC,p);
    draw(D--M--t.A--D);
    point EE=intersectionpoint(t.BC,p);
    label("$E$",EE,NE);
    label("$D$",D,E);
    draw(t.C--M);

    markrightangle(t.B,t.C,D, size=2mm);
    markrightangle(D,M,t.B, size=2mm);
    
\end{asy}

\pagebreak

\item In Diagram below, ABFE, BCGF and CDHG are all squares.
\begin{enumerate}
    \item Prove that $\triangle{AFG}\sim\triangle{HFA}$
    \item What is $\angle{AGF}+\angle{AHG}$?
\end{enumerate}

\begin{asy}
    import geometry;
    size(8cm);
    point[] pt;
    point[] pb;

    for (int i=0;i<16;i+=4) {
        pt.push((i,4));
        pb.push((i,0));
    }
    draw(pt[0]--pt[3]--pb[3]--pb[0]--cycle);
    draw(pt[1]--pb[1]);
    draw(pt[2]--pb[2]);
    draw(pt[0]--pb[1]);
    draw(pt[0]--pb[2]);
    draw(pt[0]--pb[3]);
    string[] lt=new string[]{"A","B","C","D"};
    string[] lb=new string[]{"E","F","G","H"};
    for(int i=0;i<4;i+=1) {
        label("$"+lt[i]+"$", pt[i],N);
        label("$"+lb[i]+"$", pb[i],S);
    }
\end{asy}

\item $\spadesuit$ In $\triangle{ABC}$ below, D is the midpoint of AB, if $CF:AF=1:3$, what is $CE:ED$?

\begin{asy}
    import geometry;
    size(5cm);
    triangle t=triangleabc(4,5.5,5);
    draw(t);label(t);
    
    point D=midpoint(t.AB);
    label("$D$",D,S);
    point F=(t.A+3*t.C)/4;
    point EE=intersectionpoint(line(t.C,D),line(t.B,F));
    label("$E$",EE,E);
    draw(t.C--D);
    draw(t.B--F);
    label("$F$",F,W);
\end{asy}


\item In the square ABCD below, $AE:ED=DF:FC=1:3$, prove that $\angle{DEF}=\angle{CAF}$.

\begin{asy}
    import geometry;
    size(5cm);
    point A=(0,0), B=(4,0),C=(4,4),D=(0,4),EE=(0,1),F=(1,4);
    draw(A--B--C--D--A--C);
    draw(EE--F--A);
    label("$A$",A,S);
    label("$B$",B,S);
    label("$C$",C,N);
    label("$D$",D,N);
    label("$E$",EE,W);
    label("$F$",F,N);
    point G=projection(line(A,C))*F;
    draw(F--G,dashed);
\end{asy}


\item In the diagram below, ABCD is a parallelogram. E is a point on the extension of AB. O is
the intersection of AC and BD. OE intersects BC at F. $AD=4, AB=5, BE=3$. What is the length of BF?

\begin{asy}
    import geometry;
    size(5cm);
    point A=(0,0), B=(5,0), C=(6,sqrt(15)), D=(1,sqrt(15));
    draw(A--B--C--D--A);
    point EE=(8,0);
    draw(D--B);
    draw(A--C);
    point O=intersectionpoint(line(B,D),line(A,C));
    draw(O--EE--B);
    label("$A$",A,S);
    label("$B$",B, S);
    label("$E$",EE, S);
    label("$C$",C,N);
    label("$D$",D,N);
    label("$O$",O,N);
    draw("5",A--B);
    draw("4",A--D,W);
    draw("$3$",B--EE);
    point F=intersectionpoint(line(O,EE),line(B,C));
    label("$F$",F,NE);
\end{asy}

\end{enumerate}

\end{document}
