\documentclass[letterpaper,12pt]{article}
\author{Shawn Ma}
\date{\today}
\title{Similar triangle 2}

\usepackage[pdfusetitle]{hyperref}
\usepackage{amsmath}
\usepackage{amssymb}
\usepackage{asymptote}
%\usepackage{floatrow}
\usepackage{geometry}
%\usepackage{multicol}

\geometry{left=2cm,right=2cm,top=1.5cm,bottom=1.5cm}
\begin{document}
\setlength{\parindent}{0pt}
%\renewcommand{\familydefault}{\sfdefault}

\section{Review}

Please see last week's handouts for the concepts that are required for this week's practice.

\section{Practice}
\begin{enumerate}

\item In $\triangle{ABC}$ below, $\angle{CAB}=\angle{CED}$. Prove that
\begin{enumerate}
    \item $\triangle{CAB}\sim\triangle{CED}$
    \item $\triangle{CDB}\sim\triangle{CEA}$
    \item $\triangle{DOA}\sim\triangle{EOB}$
    \item $\triangle{EOD}\sim\triangle{BOA}$
\end{enumerate}

\begin{asy}
    import geometry;
    size(5cm);
    triangle t=triangleabc(6,8,7);
    draw(t);label(t);

    point EE=(2*t.B+t.C)/3;

    real b=degrees(t.AB,t.AC);
    line k=rotate(b, EE)*line(t.C, EE);
    point D=intersectionpoint(k, t.AC);

    draw(t.A--EE);draw(t.B--D);draw(D--EE);
    point O=intersectionpoint(line(t.B,D), line(t.A,EE));
    label("$D$",D,W);
    label("$O$",O,S);
    label("$E$",EE,E);
    markangle(t.B,t.A,t.C, radius=5mm);
    markangle(t.C,EE,D, radius=5mm);
\end{asy}


\item In $\triangle{ABC}$ below, $BD\perp AC, AE\perp BE$.
\begin{enumerate}
    \item Prove that $\triangle{CBD}\sim\triangle{CAE}$
    \item Prove that $\triangle{CDE}\sim\triangle{CBA}$
    \item If $\angle{C}=60^\circ$, what is $\cfrac{DE}{AB}$?
\end{enumerate}

\begin{asy}
    import geometry;
    size(5cm);
    triangle t=triangleabc(5,6,6);
    draw(t);label(t);

    point EE=foot(t.VA);
    point D=foot(t.VB);
    draw(t.A--EE);draw(t.B--D);draw(D--EE);
    point O=intersectionpoint(line(t.B,D), line(t.A,EE));
    label("$D$",D,W);
    label("$O$",O,S);
    label("$E$",EE,E);
    markrightangle(t.A,EE,t.B);
    markrightangle(t.B,D,t.A);

\end{asy}

\item Which of the following statements are not true?
\begin{enumerate}
    \item All the right triangles are similar.
    \item All the isosceles triangles are similar.
    \item All the right isosceles triangles are similar.
    \item All the equilateral triangles are similar.
\end{enumerate}

\pagebreak

\item In the diagram below, $\triangle{ACB}$ and $\triangle{ADC}$ are right triangles.
E is the midpoint of AB. AC bisects $\angle{DAB}$.
\begin{enumerate}
    \item Prove that $AC^2=AB\cdot{AD}$
    \item Prove that $\triangle{AFD}\sim\triangle{CFE}$ (Note: the median of hypotenuse equals to half of the hypotenuse.)
\end{enumerate}

\begin{asy}
    import geometry;
    size(5cm);
    triangle t=triangleabc(3,4,5);
    draw(t);label(t);
    triangle t2=rotate(degrees(t.AB,t.AC))*scale(4/5)*t;
    draw(t2);
    label(t2, LC="$D$",LA="",LB="");
    point EE=midpoint(t.AB);
    draw(t2.C--EE--t.C);
    label("$E$",EE,S);

    point F=intersectionpoint(line(t2.C,EE),t.AC);
    label("$F$",F,W);

    markrightangle(t.A,t.C,t.B);
    markrightangle(t.A,t2.C,t.C);

\end{asy}

\item In the diagram below, $CE:BE=1:2$. $BD\parallel{AC}$. The area of $\triangle{ACE}=8$,
What is the area of $\triangle{ABD}$? (Recall the areas we learned two weeks ago.)

\begin{asy}
    import geometry;
    size(5cm);
    triangle t=triangleabc(6,5,6);
    draw(t);label(t);

    point EE=(t.C*2+t.B)/3;
    line p=parallel(t.B,t.AC);
    point D=intersectionpoint(p,line(t.A, EE));
    draw(t.B--D--t.A);
    label("$D$",D,E);
    label("$E$",EE,E);

\end{asy}

\item In the diagram below, $\triangle{CBA}\sim\triangle{CED}$. Prove that $\triangle{CAD}\sim\triangle{CBE}$.

\begin{asy}
    import geometry;
    size(5cm);
    triangle t=triangle((0,0),(3,1),(4,5));
    draw(t);label(t);

    triangle t2=rotate(50,t.C)*scale(1.3,t.C)*t;
    draw(t2);
    label(t2, LC="",LB="$E$",LA="$D$");
    draw(t.B--t2.B);draw(t.A--t2.A);

\end{asy}


\item In the diagram below, ABCD is a parallelogram. $AE:BE=1:2$. Find the ratio of the
perimeter of $\triangle{AEF}$ and $\triangle{CDF}$. If $[\triangle{AEF}]=6$, what is the area of $\triangle{CDF}$?

\begin{asy}
    import geometry;
    size(5cm);
    point A=(0,0), B=(3,0), C=(4,3), D=(1,3);
    draw(A--B--C--D--A);
    point EE=(1,0);
    draw(D--EE);
    draw(A--C);
    label("$A$",A,S);
    label("$B$",B, S);
    label("$E$",EE, S);
    label("$C$",C,N);
    label("$D$",D,N);
    point f=intersectionpoint(line(D,EE),line(A,C));
    label("$F$",f,E);

\end{asy}

\end{enumerate}

\end{document}
