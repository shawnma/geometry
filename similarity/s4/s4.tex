\documentclass[letterpaper,12pt]{article}
\author{Shawn Ma}
\date{\today}
\title{Similar triangle 3}

\usepackage[pdfusetitle]{hyperref}
\usepackage{amsmath}
\usepackage{amssymb}
\usepackage{asymptote}
%\usepackage{floatrow}
\usepackage{geometry}
%\usepackage{multicol}

\geometry{left=2cm,right=2cm,top=1.5cm,bottom=1.5cm}
\begin{document}
\setlength{\parindent}{0pt}
%\renewcommand{\familydefault}{\sfdefault}

\begin{enumerate}

\item In trapezoid below, F is the midpoint of $\overline{BC}$. $EF\parallel{AB}$. If AB=6, EF=5, what is the length of CD?

Follow up question: What if F is not the midpoint, but $CF:BF=2:1$?

\begin{asy}
    import geometry;
    size(4cm);
    point A=(0,0), B=(4,0), C=(3.3,3), D=(1,3);
    draw(A--B--C--D--cycle);
    label("$A$",A,S);
    label("$B$",B,S);
    label("$C$",C,N);
    label("$D$",D,N);

    point EE=(A+D)/2, F=(B+C)/2;

    label("$E$",EE,W);
    label("$F$",F,E);
    draw(EE--F);
\end{asy}

\item In the square ABCD below, M is the midpoint of BC, AP:PB = 3:1. Prove that \\ $\triangle{MBP}\sim\triangle{DCM}$.

\begin{asy}
    import geometry;
    size(4cm);
    point A=(0,0), B=(4,0), C=(4,4), D=(0,4);
    draw(A--B--C--D--cycle);
    label("$A$",A,S);
    label("$B$",B,S);
    label("$C$",C,N);
    label("$D$",D,N);
    point M=midpoint(line(B,C));
    point P=(A+B*3)/4;
    draw(D--M--P);
    label("$P$",P,S);
    label("$M$",M, E);
\end{asy}

% 3
\item In the diagram below, $\angle{ABC}=\angle{CDB}=90^\circ. BC=\sqrt{6}, CD=2$. What is the length of AC if $\triangle{ABC}\sim\triangle{CDB}$?

\begin{asy}
    import geometry;
    size(5cm);
    triangle t=triangleabc(3,5,4);
    draw(t);label(t);

    line k=rotate(-degrees(t.AC,t.BC),t.B)*t.BC;
    point D=projection(k)*t.C;
    draw(t.B--D--t.C);
    label("$D$",D,E);
    
\end{asy}

%4 
\item In Diagram below, OP is a light pole and height is 30ft. A person with height 5 ft walks from point A to B. AO=40ft, BO=15ft. Does the person's shadow
get shorter or longer? by how much?

\begin{asy}
    import geometry;
    size(5cm);
    point O=(0,0), P=(0,4), M=(7,0),Nx=(3,0);
    line i=line((0,0.5), (1,0.5));
    point B1=intersectionpoint(i, line(P,Nx));
    point A1=intersectionpoint(i, line(P,M));
    line x=line((0,0),(1,0));
    point B=projection(x)*B1;
    point A=projection(x)*A1;
    pen xp=linewidth(3*bp);
    draw(O--P, xp);
    draw(P--M, dashed);
    draw(P--Nx, dashed);
    draw(A1--A,xp);
    draw(B1--B,xp);
    draw(O--M);
    label("$O$",O,S);
    label("$P$",P,N);
    label("$M$",M,S);
    label("$N$",Nx,SE);
    label("$A$",A,S);
    label("$B$",B,S);
\end{asy}

\pagebreak
\item In $\triangle{ABC}$ below, $\angle{CAB}=\angle{CED}$. Prove that $\triangle{DOA}\sim\triangle{EOB}$.

\begin{asy}
    import geometry;
    size(5cm);
    triangle t=triangleabc(6,8,7);
    draw(t);label(t);

    point EE=(2*t.B+t.C)/3;

    real b=degrees(t.AB,t.AC);
    line k=rotate(b, EE)*line(t.C, EE);
    point D=intersectionpoint(k, t.AC);

    draw(t.A--EE);draw(t.B--D);draw(D--EE);
    point O=intersectionpoint(line(t.B,D), line(t.A,EE));
    label("$D$",D,W);
    label("$O$",O,S);
    label("$E$",EE,E);
    markangle(t.B,t.A,t.C, radius=5mm);
    markangle(t.C,EE,D, radius=5mm);
\end{asy}

\item In the diagram below, $CE:BE=1:2$. $BD\parallel{AC}$. The area of $\triangle{ACE}=1$,
What is the area of $\triangle{ABD}$?

\begin{asy}
    import geometry;
    size(5cm);
    triangle t=triangleabc(6,3,4);
    draw(t);label(t);

    point EE=(t.C*2+t.B)/3;
    line p=parallel(t.B,t.AC);
    point D=intersectionpoint(p,line(t.A, EE));
    draw(t.B--D--t.A);
    label("$D$",D,N);
    label("$E$",EE,NE);

\end{asy}

\item In the triangle below, $\angle{BCD}=\angle{A}$, CD=6, AC=8, AD=7, what is the length of BD?

\begin{asy}
    import geometry;
    size(5cm);
    triangle t=triangleabc(2.5,3.5,4);
    draw(t);label(t);
    line k=rotate(degrees(t.AC,t.AB), t.C)*t.BC;
    point D=intersectionpoint(k, t.AB);
    draw(t.C--D);
    label("$D$",D,S);

\end{asy}

\item In the rectangle ABCD below, E is the midpoint of BC, $DF\perp{AE}$. AD=10, AB=12, what is the length of DF?

\begin{asy}
    import geometry;
    size(4cm);
    point A=(0,0), B=(5,0), C=(5,4), D=(0,4);
    draw(A--B--C--D--cycle);
    label("$A$",A,S);
    label("$B$",B,S);
    label("$C$",C,N);
    label("$D$",D,N);
    point M=midpoint(line(B,C));
    point P=projection(line(A,M))*D;
    draw(A--M);
    draw(D--P);
    label("$F$",P,NE);
    label("$E$",M, E);
\end{asy}

\end{enumerate}

\end{document}
