\documentclass[letterpaper,12pt]{article}
\author{Shawn Ma}
\date{\today}
\title{Similar triangle 1}

\usepackage[pdfusetitle]{hyperref}
\usepackage{amsmath}
\usepackage{amssymb}
\usepackage{asymptote}
%\usepackage{floatrow}
\usepackage{geometry}
%\usepackage{multicol}

\geometry{left=2cm,right=2cm,top=1.5cm,bottom=1.5cm}
\begin{document}
\setlength{\parindent}{0pt}
%\renewcommand{\familydefault}{\sfdefault}

\section{Review}

\begin{enumerate}
    \item If $\triangle{ABC}\sim\triangle{A'B'C'}$, then $\angle{A}=\angle{A'},$
          $\angle{B}=\angle{B'}, \angle{C}=\angle{C'}. \cfrac{AB}{A'B'}=\cfrac{AC}{A'C'}=\cfrac{BC}{B'C'}$
    \item A lot of times, it boils down to ratio problems. For example,
        \begin{enumerate}
            \item If $\cfrac{AB}{CD}=\cfrac{EF}{GH}$, then $\cfrac{AB+EF}{CD+GH}=\cfrac{AB}{CD}$.
            \item If $\cfrac{AB}{CD}=\cfrac{EF}{GH}$, then $\cfrac{AB}{EF}=\cfrac{CD}{GH}$.
        \end{enumerate}
    \item Common scenarios
        \begin{enumerate}
            \item SSS similarity
            \item SAS similarity
            \item AA similarity
        \end{enumerate}
    \item Additional Theorems
        \begin{enumerate}
            \item A line is parallel to a side of a triangle, the triangle formed by the other two sides and the line
            is similar to the original one.

            \begin{asy}
                import geometry;
                defaultpen(1);
                size(5cm);
                point A=(-4,0),B=(3,0),C=(1,5),D,EE,F;
                line p=parallel((0,2),line(A,B));
                D=intersectionpoint(p, line(A,C));
                EE=intersectionpoint(p, line(B,C));
                draw(A--B--C--A);
                draw(D--EE, red*0.8);
                label("$A$",A,W);
                label("$B$",B,E);
                label("$C$",C,N);
                label("$D$",D,W);
                label("$E$",EE,E);
            \end{asy}
            \begin{asy}
                import geometry;
                defaultpen(1);
                size(5cm);
                point A=(-4,0),B=(3,0),C=(1,3),D,EE,F;
                line p=parallel((0,5),line(A,B));
                D=intersectionpoint(p, line(A,C));
                EE=intersectionpoint(p, line(B,C));
                draw(A--B--C--A--D);
                draw(EE--C);
                draw(D--EE, red*0.8);
                label("$A$",A,W);
                label("$B$",B,E);
                label("$C$",C,W);
                label("$D$",D,E);
                label("$E$",EE,W);
            \end{asy}

            $DE\parallel{AB}, \triangle{CDE}\sim\triangle{ABC}$.

            \item If a line bisects an angle of a triangle, then it divides the opposite
             side into segments whose lengths are proportional to the lengths of the other two sides.

             \begin{asy}
                import geometry;
                size(7cm);
                defaultpen(1);
                triangle t=triangleabc(3,4,3.5);
                line p=bisector(t.VA);
                point D=intersectionpoint(p, line(t.B,t.C));
                draw(t);
                label(t);
                draw(t.A--D, red*0.8);
                label("$D$",D,E);
                markangle(t.B,t.A,D);
                markangle(D,t.A,t.C,radius=0.8cm);
                p=parallel(t.C,t.AB);
                point EE = intersectionpoint(p,line(t.A,D));
                draw(t.C--EE--D, linetype(new real[] {3,3}));
                label("$E$",EE,E);
            \end{asy}
            
            AD bisects $\angle{A}$, $\cfrac{AC}{DC}=\cfrac{AB}{DB}$

            \item If two triangles are similar, their corresponding bisector, corresponding median, corresponding height and perimeters are
                also proportional to their sides. The areas are proportional to their sides' square.

             \begin{asy}
                import geometry;
                size(15cm);
                defaultpen(1);
                triangle t=triangleabc(2.5,4,5);
                triangle t2=triangleabc(2.5*1.3,4*1.3,5*1.3, (7,0));
                draw(t2);label(LA="$A'$",LB="$B'$", LC="$C'$", t2);
                draw(t);label(t);
                line p=bisector(t.VC);
                point D=intersectionpoint(p, t.AB);
                draw(t.C--D, red*0.8);
                label("$D$",D,S);
                p=bisector(t2.VC);
                point D2=intersectionpoint(p, t2.AB);
                draw(t2.C--D2, red*0.8);
                label("$D'$",D2,S);

                point H=foot(t.VC);
                draw(t.C--H, blue*0.8);
                label("$H$",H,S);
                H=foot(t2.VC);
                draw(t2.C--H, blue*0.8);
                label("$H'$",H,S);

                point M=midpoint(t.AB);
                draw(t.C--M, green*0.8);
                label("$M$",M,S);
                M=midpoint(t2.AB);
                draw(t2.C--M, green*0.8);
                label("$M'$",M,S);
            \end{asy}
            
            $\triangle{ABC}\sim\triangle{A'B'C'}$, $CM$ is the median, $CD$ is the bisector, $CH$ is the height.
            $\cfrac{CM}{C'M'}=\cfrac{CD}{C'D'}=\cfrac{CH}{C'H'}=\cfrac{AB}{A'B'}=\cfrac{P_{ABC}}{P_{A'B'C'}}$.
            $\cfrac{[ABC]}{[A'B'C']}=\cfrac{AB^2}{A'B'^2}$

        \end{enumerate} 
\end{enumerate}

\section{Practice}
\begin{enumerate}

\item Given that $\cfrac{x}{3}=\cfrac{y}{4}=\cfrac{z}{6}$, find $\cfrac{x+y+z}{x+y-z}$.
\item Prove all the theorems above. (4.a, 4.b, 4.c)
\item In $\triangle{ABC}$ below, D,E are midpoints of AC and BC. AE and BD intersect at O. Prove that $OE\times{OB}=OD\times{OA}$.

\begin{asy}
    import geometry;
    defaultpen(1);
    size(5cm);
    triangle t=triangleabc(5,6,7);
    line p=parallel((0,2),t.AB);
    point D=intersectionpoint(p, t.AC);
    point EE=intersectionpoint(p, t.BC);
    draw(t);label(t);
    draw(t.A--EE);draw(t.B--D);
    point O=intersectionpoint(line(t.B,D), line(t.A,EE));
    label("$D$",D,W);
    label("$O$",O,S);
    label("$E$",EE,E);
\end{asy}

\item In the trapezoid ABCD below, $EF\parallel{AB}$, prove that $\cfrac{DE}{EA}=\cfrac{CF}{FB}$.

\begin{asy}
    import geometry;
    defaultpen(1);
    size(5cm);
    point A=(-4,0),B=(3,0),C=(1,5),D=(-3,5),EE,F;
    line p=parallel((0,2),line(A,B));
    EE=intersectionpoint(p, line(A,D));
    F=intersectionpoint(p, line(B,C));
    draw(A--B--C--D--A);
    draw(EE--F);
    label("$A$",A,W);
    label("$B$",B,E);
    label("$C$",C,E);
    label("$D$",D,W);
    label("$E$",EE,W);
    label("$F$",F,E);
\end{asy}

\item In $\triangle{ABC}$ below, $\angle{A}=\angle{B}=\angle{DCE}$. Prove that 
\begin{enumerate}
    \item $\triangle{ACE}\sim\triangle{BDC}.$
    \item If $\angle{ACB}=90^\circ$, prove that $AB^2=2\cdot AE\cdot BD$.
\end{enumerate}

\begin{asy}
    import geometry;
    defaultpen(1);
    size(5cm);
    triangle t=triangleabc(5,5,5*sqrt(2));
    point D=(1.5,0);
    draw(t);label(t);
    draw(t.C--D);
    label("$D$",D,S);
    real b=degrees(t.AB,t.AC);
    line k=rotate(b, t.C)*line(t.C, D);
    point EE=intersectionpoint(k, t.AB);
    draw(t.C--EE);
    label("$E$",EE,S);
\end{asy}

\end{enumerate}

\end{document}
