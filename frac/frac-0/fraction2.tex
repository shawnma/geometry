\documentclass[12pt,letterpaper]{article}
\author{Shawn Ma}
\date{\today}

\usepackage{amsmath}
\usepackage{amssymb}
\usepackage[letterspace=150]{microtype}
%\usepackage{asymptote}
\usepackage{geometry}
\usepackage{enumitem}
\usepackage{multicol}
\usepackage{setspace}
\usepackage{wasysym}
\usepackage{tcolorbox}

\geometry{left=2cm,right=2cm,top=1.5cm,bottom=1.5cm}
\begin{document}
%不要自动缩进
\setlength{\parindent}{0pt}
\setstretch{1.25}

\begin{large}
\begin{enumerate}
    \item Write each fraction below as a mixed number in simplest form.
    \begin{multicols}{2}
    \begin{enumerate}
        \item $\cfrac{53}{10}=$
        \item $\cfrac{70}{9}=$
        \item $\cfrac{35}{4}=$
        \item $\cfrac{34}{8}=$
        \item $\cfrac{92}{6}=$
        \item $\cfrac{68}{12}=$
        \item $\cfrac{137}{5}=$
        \item $\cfrac{119}{14}=$
    \end{enumerate}
    \end{multicols}

    \item How many whole numbers are between $\dfrac{39}{7}$ and $\dfrac{44}{3}$?\\\\
    \item Circle the fraction below that is closest to 10.
    \[
\dfrac{79}{5}\hspace{1cm}
\dfrac{17}{11}\hspace{1cm}
\dfrac{111}{8}\hspace{1cm}
\dfrac{51}{4}\hspace{1cm}
\]

    \item Write the four fractions below in order from least to greatest.
    \[
        \dfrac{59}{9}\hspace{1cm}
        \dfrac{27}{5}\hspace{1cm}
        \dfrac{45}{11}\hspace{1cm}
        \dfrac{31}{8}\hspace{1cm}
        \]
    \rule[-1cm]{\textwidth}{1pt}
      \item Between which two consecutive whole numbers
      is $\dfrac{50}{6}+\dfrac{65}{7}$?\\\\
     
     \item Place $<$ or $>$ in the circle to compare the expressions below.
\[
\dfrac{33}{4}+\dfrac{23}{7}\bigcirc\dfrac{15}{2}+\dfrac{33}{16}
\]

\pagebreak

\item  Write each mixed number as a fraction in simplest form.
\begin{multicols}{2}
    \begin{enumerate}
\item $5\dfrac{1}{2}=$
\item $7\dfrac{2}{3}=$
\item $11\dfrac{6}{7}=$
\item $4\dfrac{4}{10}=$
\item $6\dfrac{4}{5}=$
\item $9\dfrac{4}{9}=$
    \end{enumerate}
\end{multicols}
    
\item Count by 11ths starting from $\cfrac{1}{11}$.
\newcommand\answerbox{%%
    \fbox{\rule{0.7cm}{0pt}\rule[-2ex]{0pt}{6ex}}}
\[
\cfrac{1}{11},\cfrac{2}{11},\cfrac{3}{11},\answerbox,\answerbox,\answerbox,\answerbox
\]

\item Complete the skip-counting pattern below.
\[
\cfrac{3}{12},\cfrac{5}{12},\cfrac{7}{12},\answerbox,\answerbox,\answerbox,\answerbox
\]

\begin{tcolorbox}[title=PRACTICE]
    Fill in the missing numbers in each skip-counting pattern below. Write each number in simplest form, using whole or mixed numbers when possible.
\end{tcolorbox}

\item \[
    \cfrac{2}{35},\cfrac{3}{35},\cfrac{4}{35},\answerbox,\answerbox,\answerbox,\answerbox
    \]
    \item \[
        \cfrac{1}{6},\cfrac{1}{3},\answerbox,\answerbox,\cfrac{5}{6},\answerbox,\answerbox
        \]
        \item \[
    \cfrac{1}{3},\cfrac{3}{5},\cfrac{13}{15},\answerbox,\answerbox,\answerbox,\answerbox
    \]
\end{enumerate}
\end{large}
\end{document}