\documentclass[letterpaper,12pt]{article}
\author{Shawn Ma}
\date{\today}
\title{Congurency, practice 7}

\usepackage[pdfusetitle]{hyperref}
\usepackage{amsmath}
\usepackage{amssymb}
\usepackage{asymptote}
%\usepackage{floatrow}
\usepackage{geometry}
\usepackage{multicol}

\geometry{left=2cm,right=2cm,top=1.5cm,bottom=1.5cm}
\begin{document}
\setlength{\parindent}{0pt}

\begin{enumerate}
\item Express each difference in simplest form, using mixed numbers if possible.
\begin{multicols}{2}

\begin{enumerate}
    \item $4\dfrac{7}{15}-2\dfrac{14}{15}=$
    \item $6\dfrac{7}{18}-2\dfrac{13}{18}=$
    \item $5\dfrac{3}{8}-1\dfrac{7}{8}=$
    \item $12\dfrac{8}{13}-4\dfrac{10}{13}=$
    \item $9\dfrac{19}{21}+3\dfrac{20}{21}=$
    \item $5\dfrac{7}{12}-2\dfrac{11}{12}=$
    \item $4\dfrac{7}{9}+2\dfrac{5}{9}=$
    \item $2\dfrac{16}{17}+\dfrac{1}{17}=$
\end{enumerate}
\end{multicols}

\item Write the perimeter of each shape below as a whole or mixed number in simplest form. Remember to include units.
\begin{multicols}{2}
\begin{enumerate}
    \item Perimeter =

    \begin{asy}
        import markers;
        import math;
        import geometry;
        size(5cm);
        pair A,B,C,D,EE,F,G,H;
        A=(1.5,4);
        B=(-3,0);
        C=(3,0);
        draw(A--B--C--cycle);
        label("$5\frac{5}{8}in$",A--B, NW);
        label("$3\frac{3}{8}in$",A--C, NE);
        label("$7\frac{7}{8}in$",C--B, S);
    \end{asy}
    \item Perimeter =

    \begin{asy}
        import markers;
        import math;
        import geometry;
        size(5cm);
        pair A,B,C,D,EE,F,G,H;
        A=(0,2.3);
        B=(0,0);
        C=(5.7,0);
        D=(5.7,2.3);
        draw(A--B--C--D--cycle);
        label("$2\frac{3}{10}cm$",A--B,E );
        label("$5\frac{7}{10}cm$",A--D, N);
    \end{asy}
\end{enumerate}
\end{multicols}


\item Emma runs 10 days in a row. On the first day, she runs $\dfrac{5}{6}$ miles. Each day after the first, she runs $\dfrac{1}{6}$ miles farther
than she did the day before. How many miles does Emma run all together?

\item The width of the rectangle below is twice its height. The perimeter of the rectangle
is 21cm. What is its height in centimeters?

\begin{asy}
    import markers;
    import math;
    import geometry;
    size(5cm);
    pair A,B,C,D,EE,F,G,H;
    A=(0,2);
    B=(0,0);
    C=(4,0);
    D=(4,2);
    draw(A--B--C--D--cycle);
\end{asy}

\item Two equilateral triangles are arranged as shown below to create
a rhombus. The perimeter of each triangle is 8 inches. What is the perimeter
of the rhombus in inches?

\begin{asy}
    import markers;
    import math;
    import geometry;
    size(5cm);
    pair A,B,C,D,EE,F,G,H;
    A=(0,2*sqrt(3));
    B=(-2,0);
    C=(2,0);
    D=(4,2*sqrt(3));
    draw(A--B--C--D--cycle);
    draw(A--C);
\end{asy}

\item Luke is $5\frac{1}{4}$ inches taller than Jess. The sum of Luck's and
Jess's heights is $82\frac{1}{4}$ inches. How many inches tall is Jess?
\end{enumerate}

\end{document}