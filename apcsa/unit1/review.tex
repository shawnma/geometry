\documentclass[aspectratio=169]{beamer}
\usepackage{listings}
\usetheme{metropolis}
\usepackage{xcolor}
\usepackage{newverbs}

\renewenvironment{verbatim}
{\semiverbatim\color{red}}
{\endsemiverbatim}
\renewcommand{\verb}{\collectverb{\color{codegreen}}}

\usepackage{inconsolata}
\usepackage[T1]{fontenc}

\definecolor{codegreen}{rgb}{0,0.6,0}
\definecolor{codegray}{rgb}{0.5,0.5,0.5}
\definecolor{codepurple}{rgb}{0.58,0,0.82}
\definecolor{backcolour}{rgb}{0.95,0.95,0.92}

\lstdefinestyle{mystyle}{
    commentstyle=\color{codegreen},
    keywordstyle=\color{magenta},
    numberstyle=\tiny\color{codegray},
    stringstyle=\color{codepurple},
    basicstyle=\ttfamily\footnotesize,
    backgroundcolor=\color{backcolour},
    breaklines=true,
    showstringspaces=false,
    captionpos=b,                    
    tabsize=2
}

\lstset{style=mystyle}

\title{AP CS A\\Unit 1 - Getting Started and Primitive Types}
\date{\today}
\author{Shawn Ma}

\begin{document}
    \maketitle
    \begin{frame}[fragile]{Java Program structure}
   
    \begin{lstlisting}[language=Java]
public class MyClass {
    public static void main(String[] args) {
        System.out.println("hello, world!");
    }
}
        \end{lstlisting}
    \begin{itemize}
        \item The curly braces \alert{\{} could be at the end of the line or new line.
        \item The class name must be the same as the file name (with a \verb|java| suffix.)
        \item Java is case sensitive.
        \item Semi-colon is required.
    \end{itemize}

\end{frame}
\begin{frame}{Print to console}
\begin{itemize}
    \item \alert{System.out.println} prints the value followed by a new line (ln)
    \item \alert{System.out.print} doesn't advance to the next line.

\end{itemize}
\end{frame}

\begin{frame}[fragile]{Comments and Strings}
\begin{itemize}
    \item \alert{Comments} has two forms. \verb|//| and \verb|/* */|
    \item \verb|/* */| could span multiple rows.
    \item A string literal is enclosed in double quotes \verb|" "|.
\end{itemize}
\end{frame}

\begin{frame}[fragile]{Compilation and Execution}
\begin{itemize}
    \item Java code is firstly \alert{compiled} into something called \verb|byte code|,
        then the Java Virtual Machine (JVM) will load the code and \alert{execute} it.
        \begin{lstlisting}[language=bash]
$ javac Myclass.java # Compile; your IDE may be doing this for you
$ ls
MyClass.java  MyClass.class
# Execution; the java command runs JVM and finds the MyClass.class and its main method
$ java MyClass
hello world!
            \end{lstlisting}
    \item This is called \alert{compiled} language. (C++, golang are another examples)
    \item Python, Javascript are compiled and executed at the same time, they are called \alert{interpreted} languages.
\end{itemize}
\end{frame}

\begin{frame}[fragile]{Variable and Data Type}
\begin{itemize}
    \item A variable is a name for a memory location where you can store a value that can \alert{change or vary}. In Java, they are mostly \verb|camelCase|.
    \begin{itemize}
        \item On the contrast, invariant objects are called \verb|constants| or often \verb|const|, in Java, it has a modifier of \verb|final| and the variable names are typically \verb|ALL_CAP_CASES|.
    \end{itemize}
    \item Command data Types
    \begin{itemize}
        \item \verb|int|
        \item \verb|double|
        \item \verb|boolean|
        \item \verb|String|
    \end{itemize}
    \item Variables can be declared and initialized (with a \verb|=|) at the same time.
    \begin{itemize}
        \item Uninitialized variables get a default value. (0 for primitive types, null for Objects)
    \end{itemize}
    \item \verb|int| and \verb|double| are primitive types and \verb|String| is an object.
\end{itemize}
\end{frame}

\begin{frame}[fragile]{Primitive vs Object}
    \begin{itemize}
        \item Primitive types include only \verb|int|, \verb|double|, \verb|char|, \verb|boolean| etc (total 8). Everything else is an object.
        \item The key difference is primitives are passed by \alert{values} and objects are passed by \alert{reference}.
        \item Objects have to be initialized with a \verb|new| keyword.
        \begin{lstlisting}[language=Java,basicstyle=\small,style=mystyle]
public void addOne(int x) {
    x++;
}
int y = 1;
addOne(y);
System.out.println(y);
\end{lstlisting}
    \end{itemize}
    We'll cover Objects mutability later.
\end{frame}

\begin{frame}[fragile]{Operators}
    \begin{itemize}
        \item Arithmetic operators \verb|+, -, *, /, %|
        \item Unary operators \verb|-, ++, --, !|
        \item Compound operators \verb|+=, -=, *=, /=, %=|
        \item Equality and Relational Operators \verb|>, <, ==, >=, <=, !=|
        \item Conditional Operators \verb_||, &&, ?:_   
        \item \verb|^| is bit operation (XOR); exponent is done through \verb|Math.pow()|
        \item Prefix \& Postfix operator of \verb|-- & ++|
    \end{itemize}
\end{frame}

\begin{frame}[fragile]{Operators}
    \begin{itemize}
        \item \verb|/| is special for integers. It also doesn't round. To round, use \verb|double| type and \verb|Math.round()|
        \item \verb|==| compares objects or primitive equality, however, for objects, it only compares the reference to the objects.
        \begin{itemize}
            \item To compare the content equality of objects, use \verb|.equals()| method, e.g., \verb|"text".equals(myString)|
        \end{itemize}
        \item In a lot of times, \verb|double| couldn't be compared with \verb|==| due to the precision loss of Java.
        \item Divide by \verb|0| will cause a Runtime Exception called \verb|ArithmeticException|
    \end{itemize}
\end{frame}

\begin{frame}[fragile]{Read user input}
    \begin{lstlisting}[language=Java]
import java.util.Scanner;

public class Main {
  public static void main(String[] args) {
    System.out.println("Please type in a name in the input box below.");
    Scanner scan = new Scanner(System.in);
    String name = scan.nextLine();
    System.out.println("Hello " + name);
    scan.close();
  }
}
        
    \end{lstlisting}
\end{frame}

\begin{frame}[fragile]{Casting}
    \begin{itemize}
        \item Convert one type to another
        \item For now it's \verb|(int)| and \verb|(double)|
        \item \verb|(int)| truncates the double type.
        \item \verb|int|s have a much smaller range \verb|[Integer.MIN_VALUE, Integer.MAX_VALUE]|
        \item Integer computations could go out of the range and cause overflow.
        \item Objects could be casted only they are on the same inheritance tree.
    \end{itemize}
    Is \verb|x| and \verb|y| the same?
    \begin{lstlisting}[language=Java]
double x = (double) 1/3;
double y = (double) (1/3);
    \end{lstlisting}
\end{frame}


\begin{frame}[fragile]{Method Signatures}
    \begin{itemize}
        \item Method: \verb|methodName(VariableType variableName, ...)|
        \item \verb|System.out.println| is a method. It is also an \alert{overloaded} method.
        \item Concepts: Method signature, parameters, parameter types, argument        
        \item All the parameters are in fact call by \alert{values}. Call by reference is passing a copy of the reference.        
    \end{itemize}
\end{frame}

\begin{frame}[fragile]{Class Methods}
    \begin{itemize}
        \item \verb|public static return-type method-name(parameters)|
        \item Method could have a return value (it could be ignored by caller)
    \end{itemize}
\end{frame}

\end{document}