\documentclass[aspectratio=169]{beamer}
\usepackage{listings}
\usetheme{metropolis}
\usepackage{svg}
\usepackage{../slides}
\usepackage{tabularx}

\title{AP CS A\\Unit 2 - Objects}
\date{\today}
\author{Shawn Ma}

\begin{document}
\maketitle
\begin{frame}[fragile]{Class vs Object}
  \begin{itemize}
    \item Class is the \alert{blueprint} of Objects.
    \item Class defines
      \begin{itemize}
        \item attributes (instance variables)
        \item behaviors (methods)
      \end{itemize}
    \item Objects are created using \verb|new| keyword
    \begin{lstlisting}[language=Java]
    Cat cat = new Cat();
    \end{lstlisting}
      \begin{alertblock}{Pop Quiz}
        \begin{itemize}
          \item How many cat objects could we create in Java?
          \item How many Cat class do we have in Java?
        \end{itemize}
      \end{alertblock}
  \end{itemize}
\end{frame}

\begin{frame}[fragile]{Class vs Object}
  If you design a Tetris game, how do you design your classes/objects?
  \begin{center}
    \includesvg[width = 100pt]{tetris.svg}
  \end{center}
\end{frame}

\begin{frame}[fragile]{Constructor}
  \begin{itemize}
    \item Constructor is a special method of a Java class with \alert{no return value}
    \item If no Constructor is provided, a default Constructor will be "generated" which does nothing.
      \begin{itemize}
        \item so that we could still invoke it
      \end{itemize}
    \item Similar to methods, Constructors could be overloaded with different parameters.
    \begin{lstlisting}[language=Java]
class Cat {
  final String name;
  public Cat() {
    name = "Taro";
  }
  public Cat(String catName) {
    name = catName;
  }
}
Cat mycat = new Cat("spud");
System.out.println(mycat.name);
    \end{lstlisting}
  \end{itemize}
\end{frame}

\begin{frame}[fragile]{General class structure}
  \begin{lstlisting}[language=Java]
    class Cat {
      // member variables (attributes)
      private String name;

      // Constructors
      Cat() {
        this.name = "Taro";
      }

      // methods (behaviors)
      public void meow() {
        System.out.println("woof woof!");
      }
    }
  \end{lstlisting}
\end{frame}

\begin{frame}[fragile]{Referring objects}
  \begin{itemize}
    \item By default an object variable points to \verb|null| unless assigned (e.g., \verb|new|)
    \item Invoke a method of another object using \verb|.| operator. e.g, \verb|cat.meow()|
    \item Invoke a method in same class directly, or use \verb|this|.
    \item Invoke a method in \verb|parent| class uses \verb|super| (this is to be discussed later)
    \item \verb|this| could be convenient sometimes:
  \end{itemize}
  \begin{lstlisting}[language=Java]
    public Cat(String name) {
      this.name = name;
    }
  \end{lstlisting}
  Note the scope of variable here. If no qualifier is specified, a local variable will override an argument
  variable, then the member variable.
\end{frame}

\begin{frame}[fragile]{Keywords review}
  \begin{itemize}
    \item \verb|new|
    \item \verb|this| - the current class
    \item \verb|super| - parent class (more on this when we talk about inheritance)
    \item modifiers
      \begin{itemize}
        \item \verb|final| - something is constant.
        \item \verb|public, private| - visibility of a variable, method or class
        \item \verb|static| - The variable, method or class could be referenced by the class directly, without an object reference
          (global: only one copy in one Java Program)
      \end{itemize}
  \end{itemize}
  \begin{alertblock}{Pop Quiz}
    \begin{itemize}
      \item What types of modifiers should \verb|Integer.MAX_VALUE| have?
      \item What types of modifiers should System.\verb|out.print()| have?
    \end{itemize}
  \end{alertblock}
\end{frame}

\begin{frame}[fragile]{Strings}
  \begin{itemize}
    \item String is a \verb|class|.
    \item Create a new String object: \verb|String a = new String("taro");|
    \item Java implicitly creates a String object for you if we just write a String literal. It is indeed an object.
      \begin{itemize}
        \item e.g., \verb|"spud".charAt(0) == 's'|
      \end{itemize}
    \item Conveniently, Java allows us to concatenate two Strings to get a new String. It could concatenate other types as well (e.g. int, double)
      \begin{itemize}
        \item e.g., \verb|"You score = " + score| where score is an \verb|int|
        \item \alert{QUIZ:} what happens if we do \verb|"my cat is " + cat| where cat is an arbitrary object?
      \end{itemize}
    \item We could also do \verb|+=| with String.
    \item \verb|+ and +=| are only for Strings. Will not work for general objects.
    \item Since String is enclosed with double quota \verb|"|, to have a double quote within a string, we need to escape it (\verb|\"|). Now backslash also
      has a special meaning so we'll also need to escape it as well (\verb|\\|). We use it to escape other non-printable characters as well (\verb|\n, \t|)
  \end{itemize}
\end{frame}

\begin{frame}[fragile]{String methods}
  \begin{itemize}
    \item Common methods
      \begin{itemize}
        \item length()
        \item substring()
        \item indexOf()
        \item compareTo()
        \item equals()
      \end{itemize}
    \item String is immutable. Means there is no method to change the attributes of a String after creation.
  \end{itemize}
\end{frame}

\begin{frame}[fragile]{Wrapper classes}
  \begin{itemize}
    \item Integer and Double
    \item Auto-boxing (unboxing)
  \end{itemize}
\end{frame}

\begin{frame}[fragile]{Math class}
  \begin{itemize}
    \item abs()
    \item round(), floor(), ceil()
    \item pow()
    \item sqrt()
    \item random()
  \end{itemize}
\end{frame}
\end{document}
