\documentclass[aspectratio=169]{beamer}
\usepackage{listings}
\usetheme{metropolis}
\usepackage{svg}
\usepackage{../slides}
\usepackage{tabularx}

\title[Boolean Expressions]{Boolean Expressions in Java}
%\title{AP CS A\\Unit 2 - Objects}
\date{\today}
\author{Shawn Ma}

\begin{document}

\begin{frame}[fragile]{Boolean Variables and Equality Testing}
    \begin{itemize}
        \item \verb{==, !=, >=, >, <=, <}
        \item \verb{if, else, else if}
        \item \verb{&&, ||, !}
        \item Short Circuit Evaluation
    \end{itemize}
\end{frame}


\begin{frame}[fragile]{Pop Quiz}
    \begin{lstlisting}[language=Java]
int x=3, y=5;
if (++x == 4 || y++ > 5) {
    y++;
}
System.out.println(x+y);
    \end{lstlisting}
\end{frame}

\begin{frame}[fragile]{Pop Quiz}
    \begin{lstlisting}[language=Java]
boolean isOdd(int x) {
    return x%2==0;
}
boolean isPositive(int x) {
    x=Math.abs(x);
    return x > 0;
}
int x = -4;
if (isPositive(x) && !isOdd(x)) {
    System.out.println(x);
}
    \end{lstlisting}
\end{frame}

\begin{frame}[fragile]{De Morgan’s Laws}
    \begin{itemize}
        \item \verb{!(a && b)} is equivalent to \verb{!a || !b}

        \item \verb{!(a || b)} is equivalent to \verb{!a && !b}
    \end{itemize}
\end{frame}


\begin{frame}[fragile]{String equility}
    \begin{itemize}
        \item reference vs "content"
        \item null reference will cause \verb|NullPointerException|
    \end{itemize}
\end{frame}

\end{document}
