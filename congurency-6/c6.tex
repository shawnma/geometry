\documentclass[letterpaper,12pt]{article}
\author{Shawn Ma}
\title{Congurency, practice 6}
\date{\today}

\usepackage[pdfusetitle]{hyperref}
\usepackage{amsmath}
\usepackage{amssymb}
\usepackage{asymptote}
%\usepackage{floatrow}
\usepackage{geometry}
%\usepackage{multicol}

\geometry{left=2cm,right=2cm,top=2cm,bottom=2.5cm}
\begin{document}
\setlength{\parindent}{0pt}


\begin{enumerate}
\item In the diagram below, $ABCD$ is a rectangle, fold $\triangle{BCD}$ align the line of BD as $\triangle{BC'D}$. Prove that $AE=C'E$.

\begin{asy}
import markers;
import math;
import geometry;
size(5cm);
pair A,B,C,D,EE,F,G,H;
A=(0,4);
B=(0,0);
C=(6,0);
D=(6,4);
EE=(2.1,5.6);
F=extension(B,EE,A,D);
draw(A--B--C--D--A);
draw(B--D--EE--B);
label("$A$",A,N);
label("$B$",B,S);
label("$C$",C,S);
label("$D$",D,E);
label("$C'$",EE,N);
label("$E$",F,SE);
\end{asy}

\item In the diagram below, $AB=CD, \angle{ABC}=\angle{BCD}$. Prove that $OA=OD$.

\begin{asy}
import markers;
import math;
import geometry;
size(5cm);
pair A,B,C,D,EE,F,G,H;
A=(-2,4);
B=(-3,0);
C=(3,0);
D=(2,4);
draw(A--B--C--D--cycle);
draw(A--C);
draw(B--D);
F=extension(A,C,B,D);
label("$A$",A,N);
label("$B$",B,S);
label("$C$",C,S);
label("$D$",D,N);
label("$O$",F,S);
\end{asy}

\item In the right $\triangle{ACB}$ below, $CD\perp{AB}, AF$ bisects $\angle{CAB}$. Prove that $CE=CF$.

\begin{asy}
import markers;
import math;
import geometry;
size(7cm);
pair A,B,C,D,EE,F,G,H;
A=(-1,0);
B=(3,0);
C=(0,2);
D=(0,0);
G=(2,1.9);
EE=extension(C,D,A,G);
F=extension(C,B,A,G);
draw(A--B--C--cycle);
draw(A--F);
draw(C--D);
label("$A$",A,W);
label("$B$",B,S);
label("$C$",C,N);
label("$D$",D,S);
label("$E$",EE,SE);
label("$F$",F,NE);
markrightangle(A,C,B,size=2mm);
markrightangle(C,D,B,size=2mm);
\end{asy}

\item In the diagram below, right triangle $\triangle{ACB}\cong\triangle{DEB}$. Prove that $AF+EF=DE$.

\begin{asy}
import markers;
import math;
import geometry;
size(5cm);
pair A,B,C,D,EE,F,G,H;
A=(4*sqrt(3),0);
B=(0,4);
C=(0,0);
D=(4*sqrt(3),8);
F=(2,0);
EE=extension(A,B,D,F);
draw(B--C--A--B--D--F);
label("$A$",A,S);
label("$B$",B,W);
label("$C$",C,W);
label("$D$",D,E);
label("$E$",EE,E);
label("$F$",F,S);
markrightangle(D,EE,B,size=2mm);
markrightangle(A,C,B,size=2mm);
\end{asy}

\item In the diagram below, $AB=AF, BC=EF, \angle{B}=\angle{F}$, D is the midpoint of CE, prove that $AD\perp{CE}$.

\begin{asy}
import markers;
import math;
import geometry;
size(5cm);
pair A,B,C,D,EE,F,G,H;
A=(0,4);
B=(-3,2);
C=(-2.5,0);
D=(0,0);
EE=(2.5,0);
F=(3,2);
draw(A--B--C--D--EE--F--A);
draw(A--D);
label("$A$",A,N);
label("$B$",B,W);
label("$C$",C,S);
label("$D$",D,S);
label("$E$",EE,S);
label("$F$",F,E);
\end{asy}


\item $\bigstar$ In the diagram below, $AB=5, AC=3$, D is the midpoint of BC. What is the range of the length of $\overline{AD}$?

\begin{asy}
import markers;
import math;
import geometry;
size(5cm);
pair A,B,C,D,EE,F,G,H;
A=(1.5,3);
B=(-4,0);
C=(4,0);
D=(0,0);
draw(B--A--C--B);
draw(A--D);
label("$A$",A,N);
label("$B$",B,S);
label("$C$",C,S);
label("$D$",D,S);
\end{asy}


\item $\bigstar$ In the diagram below, $\angle{C}=\angle{E}, \angle{B}=\angle{F}, BC=EF$, D is the midpoint of CE, prove that $AD\perp{CE}$.

\begin{asy}
import markers;
import math;
import geometry;
size(5cm);
pair A,B,C,D,EE,F,G,H;
A=(0,4);
B=(-3,2);
C=(-2.5,0);
D=(0,0);
EE=(2.5,0);
F=(3,2);
draw(A--B--C--D--EE--F--A);
draw(A--D);
label("$A$",A,N);
label("$B$",B,W);
label("$C$",C,S);
label("$D$",D,S);
label("$E$",EE,S);
label("$F$",F,E);
\end{asy}




\item $\bigstar$ In the equalateral triangle below, $\overline{AB}=1, AE=CD, EF\perp{AC}$. What is the length of $\overline{FG}$?

\begin{asy}
import markers;
import math;
import geometry;
size(5cm);
pair A,B,C,D,EE,F,G,H,I;
A=(0, 2*sqrt(3));
B=(-2,0);
C=(2,0);
D=(3.3,0);
G=(-2,3);
H=(2,4);
EE=extension(G,D,A,B);
F=extension(A,C,EE,H);
I=extension(D,EE,A,C);
draw(B--A--C--B);
draw(C--D--EE--F);
markrightangle(EE,F,I,size=2mm);
label("$A$",A,N);
label("$B$",B,S);
label("$C$",C,S);
label("$D$",D,S);
label("$E$",EE,W);
label("$F$",F,E);
label("$G$",I,NE);
\end{asy}


\end{enumerate}


\end{document}