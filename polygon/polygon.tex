\documentclass[letterpaper,12pt]{article}
\author{Shawn Ma}
\date{\today}
\title{Polygon 1}

\usepackage[pdfusetitle]{hyperref}
\usepackage{amsmath}
\usepackage{amssymb}
\usepackage{asymptote}
%\usepackage{floatrow}
\usepackage{geometry}
%\usepackage{multicol}

\geometry{left=2cm,right=2cm,top=1.5cm,bottom=1.5cm}
\begin{document}
\setlength{\parindent}{0pt}
%\renewcommand{\familydefault}{\sfdefault}

\begin{enumerate}

\item Amy starts from point A, going straight for 12 meters then turn $36^\circ$ left, then
goes straight for 12 meters again, then turn left $36^\circ$. She repeats this until she goes
back to point A. How long has she traveled?

\begin{asy}
    import geometry;
    size(0,4cm);
    coordsys r = currentcoordsys;
    point C;
    for (int i=0; i<4; i+=1) {
        point A=point(r, (0,0));
        point B=point(r, (3,0));
        if (i!=0) {
            markangle("$36^\circ$", C,A,B, radius=6mm);
        }
        C=point(r, (6,0));
        draw(A--B);
        if (i!=3) {
            draw(B--C, dashed);
        }
        r=shift(3)*rotate(36)*r;
    }
    dot("$A$",(0,0), S);

\end{asy}

\item In the right pentagon below (left), what is $\angle{DAB}$? What about $\angle{EFB}$ in the right hexagon to the right?

\begin{asy}
    import geometry;
    size(0,4cm);
    coordsys r = currentcoordsys;
    string[] labels = {"A","B","C","D","E"};
    pair[] dir={SW,SE,E,N,W};
    real angle = 2*180/5;
    point[] vtx;
    for (int i=0; i<5; i+=1) {
        point A=point(r, (0,0));
        point B=point(r, (3,0));
        label(labels[i], A, dir[i]);
        draw(A--B);
        vtx.push(A);
        r=shift(3)*rotate(angle)*r;
    }
    draw(vtx[0]--vtx[3]);
\end{asy}
\begin{asy}
    import geometry;
    size(0,4cm);
    coordsys r = currentcoordsys;
    string[] labels = {"A","B","C","D","E","F"};
    pair[] dir={SW,SE,E,N,W,W};
    real angle = 2*180/6;
    point[] vtx;
    for (int i=0; i<6; i+=1) {
        point A=point(r, (0,0));
        point B=point(r, (3,0));
        label(labels[i], A, dir[i]);
        draw(A--B);
        vtx.push(A);
        r=shift(3)*rotate(angle)*r;
    }
    draw(vtx[1]--vtx[5]);
\end{asy}
% 3
\item In the polygon below, $\angle{ABC}=100^\circ$. What is the sum of $\angle{A}$, $\angle{C}$, $\angle{D}$ and $\angle{E}$?

\begin{asy}
    import geometry;
    size(0,4cm);
    string[] labels = {"A","B","C","D","E"};
    pair[] dir={SW,N,E,N,W};
    point[] vtx={(0,0), (3,2.2), (5,0), (4,3), (1,3.5), (0,0)};
    for (int i=0; i<5; i+=1) {
        draw(vtx[i]--vtx[i+1]);
        label(labels[i], vtx[i], dir[i]);
    }
\end{asy}

%4 
\item In the star shape below, find $\angle{A}+\angle{B}+\angle{C}+\angle{D}+\angle{E}+\angle{F}+\angle{G}+\angle{H}$.

\begin{asy}
    import geometry;
    size(0,6cm);

    string[] labels = {"A","B","C","D","E","F","G","H"};
    pair[] dir={SW,SE,E,N,N,N,W,W};
    real angle = 2*180/8;
    point[] vtx;
    coordsys r = currentcoordsys;

    for (int i=0; i<8; i+=1) {
        point A=point(r, (0,0));
        vtx.push(A);
        r=shift(2+unitrand())*rotate(angle+unitrand())*r;
    }
    for (int i=0; i<8; i+=1) {
        draw(vtx[i]--vtx[(i+3)%8]);
        draw(vtx[i]--vtx[(i+5)%8]);
        label(labels[i], vtx[i], dir[i]);
    }
    
\end{asy}

\pagebreak

\item For a polygon, after excluding one angle, the sum of the rest interior angles is $2023^\circ$. What is the size of the excluded angle and how many sides does this polygon have?

\vspace{3cm}

\item After cutting off an angle of a polygon, the sum of the interior angles of the new shape is $720^\circ$. How many sides does the original polygon have?
\vspace{3cm}

\item The sum of two convex polygons' number of sides is 12, and the sum of their number of diagonals is 19. How many sides do they have, respectively?
\vspace{3cm}

\item In the octagon below, all the interior angles are the same. The length of AB, BC, CD, DE, EF, FG are 7,4,2,5,6,2 respectively. Find the perimeter of the octagon.

\begin{asy}
    import geometry;
    size(0,6cm);
    coordsys r = currentcoordsys;
    string[] labels = {"A","B","C","D","E","F","G"};
    real[] sides={7,4,2,5,6,2,6};
    pair[] dir={SW,SE,E,N,N,W,W,W};
    real angle = 2*180/8;
    point[] vtx;
    vtx.push((0,0));
    for (int i=0; i<7; i+=1) {
        real l=sides[i];
        point A=point(r, (0,0));
        point B=point(r, (l,0));
        label(labels[i], A, dir[i]);
        draw(A--B);
        vtx.push(B);
        r= rotate(angle*(i+1), B)*cartesiansystem(B, E, N);
        //show(labels[i+1], r, xpen=invisible);
    }
    label("H",vtx[7],W);
    draw(vtx[0]--vtx[7]);
\end{asy}

\end{enumerate}

\end{document}
